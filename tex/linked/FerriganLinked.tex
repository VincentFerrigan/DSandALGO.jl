   
%! Linked Lists
%! Author = Vincent Ferrigan <ferrigan@kth.se>
%! Date = 2022-09-21


% Preamble
\documentclass[a4paper, 11pt]{article}
% Packages
\usepackage[T1]{fontenc}
\usepackage[utf8]{inputenc}
\usepackage[english]{babel}
\usepackage{fontspec}
\usepackage{polyglossia}
\usepackage{microtype}
\setmonofont{DejaVu Sans Mono}[Scale=MatchLowercase]
\usepackage{minted}
\usepackage{latexsym,exscale,stmaryrd,amsmath,amssymb}
\newtheorem{definition}{Definition}
\usepackage{unicode-math}
\usepackage{lmodern}
\usepackage{enumitem}
\usepackage{subcaption}
\usepackage{graphicx}
\usepackage{hyperref}
% \usepackage{natbib}
% \usepackage[nottoc]{tocbibind}
% \usepackage{listings}
\usepackage{xcolor}
\usepackage{siunitx}
\usepackage{tikz}
\usepackage[font=small,labelfont=bf]{caption}
% Addiding JuliaMono
\newfontfamily \JuliaMono {JuliaMono-Regular.ttf}[
    Path      = ./,
    Extension = .ttf
    ]
\newfontface \JuliaMonoMedium{JuliaMono-Regular}
\setmonofont{JuliaMonoMedium}[
    Contextuals=Alternate
]


\title{Linked lists\\ \small{ID1021 Algorithms and Data structures}} %%TODO vilken rubricering
\author{Vincent Ferrigan}

\date{\today}

\begin{document}
    \maketitle
    \section*{Introduction}
    In this study, the performance and comparison of two data structures,
    \emph{Linked Lists} and \emph{Arrays}, were studied. 
    
    This was done through benchmarking -- i.e. comparing the performance of
    the two data structures in terms of managing data. 
    By managing data we mean:
    
    \begin{enumerate}[label*=\arabic*.]
        \item The size of data
        \begin{enumerate}[label*=\arabic*.]
            \item How well do they scale?
            \item Are they dynamic or static in nature? 
        \end{enumerate}
        \item The operations on data
        \begin{enumerate}[label*=\arabic*.]
            \item Appending one to another e.g. appending one array to another
            vis-à-vis appending one linked list to another. 
            \begin{enumerate}[label*=\arabic*.]
                \item Does the type of data structure matter?
                \item Which of the lists affect the overall performance?
                Is the size of the prepending list critical or does the size of
                the appending list matter?
            \end{enumerate}
            \item Deletion/insertion. Does the position matter? That is, is
            there a difference between inserting/deleting an item in the
            beginning or end? Does the type of data structure matter?
            \item Searching for an item.
            \item Updating an item. 
        \end{enumerate}
        \item The performance of certain tasks. 
        Stack implementation is a good example. 
        Do list-based stack perform better than array-based stacks?
    \end{enumerate}
    Not all of above-mentioned will be benchmarked, however, the authors intent is to
    reach certain conclusions on the queries that were stated above. 
    This can be done through studying their implementation. 
            
    \section*{Methods}
    All the Data Structures and Algorithms were implemented in \emph{Julia}.
    The Code was mostly written in \emph{VSCode} and run on \emph{Julia 1.8.0}.
    Quick-fixes and editing was, however, done in \emph{Vim}

    Some scripts were executed from the \emph{REPL terminal},  while others (e.g.
    when using data frames, performing benchmarks and producing plots) were executed from the
    \emph{Jupyter Notebook}. 
    
    \subsection*{Tools and packages}
    All tests were performed with the built-in package \emph{Test} and iterative development
    was made possible through \emph{Revise.jl} -- the latter operates by continuously
    scanning the source code for changes, even changes in function defined in
    other modules (including in different files).
    
    The benchmark data was constructed, manipulated and visualized through
    \emph{DataFrames.jl} and \emph{Plots.jl}, while 
    readable formatting was produced through \emph{Formatting.jl} and \emph{Unitful.jl}. 

    \subsection*{The JIT}
    Julia has a just-in-time (JIT) compilation -- which means that the code is
    dynamically compiled during program run time.     
    It takes time for the JIT compiler to 
    initially load the code and compile it. Therefore, in order not to skew the
    results, \emph{warmup calls} were performed on certain parts of the code
    before they were benchmarked. This to avoid including 
    compilation time. The warmup calls were done with the @timed macro prior to
    benchmarking.

    % \section*{Learning objectives}
\clearpage

    \section*{The Data Structures}
    This section briefly describes the key differences between arrays and lists 

    \subsection*{Conventions}
    It is important for the reader to note that Julia uses a one-based-numbering
    convention, (i.e. array indices start from 1 to N) and that one dimensional
    arrays are called vectors. 
    For consistency, the author has chosen to refer to one dimensional
    arrays as vectors and apply the 1-based convention when numbering
    sequences of elements more broadly. 

    Function names that ends with a bang (!) mutates the arguments it receives. 
    This name suffix is by convention in Julia. 
    
    Unlike other languages, Julia objects cannot be ''null'' by default. The
    equivalent of \lstinline[language=Python]{None} in Python 
    or \lstinline[language=C]{NULL} and \lstinline[language=C]{void} in C is
    \mintinline{julia}{Nothing}. The Julia convention is to return the value \mintinline{julia}{nothing}, which is a
    singleton instance of type \mintinline{julia}{Nothing}, when such a side effect is
    desired. 
    \subsection*{Linked Lists}
    Linked lists are dynamic data structures that, unlike vectors, do not require
    to be stored contiguously in memory. They are instead composed of a chain of 
    nodes that are linked together by pointers or references. 
    
    The implementation of a linked list requires two data structures;
    one for the actual node and another for the linked-list, the latter containing a
    reference that points to the first node.
    Each node contains at least two parts; an item of any type and 
    the memory address of the next node in the list. 
    The last node in the list contains a next filed that holds no value.  
    -- i.e. it points to \mintinline{julia}{nothing} which, as previously mentioned, is a 
    singleton instance of type ''Nothing''. 
    Figure \ref{code:node} shows how the node can be implemented in Julia. 

    \begin{figure}[H]
        \centering
    \begin{minted}[
        label= Node Data Types (Custom Structs), 
        linenos, 
        % breaklines, 
        frame = single, 
        fontsize=\footnotesize]{julia}
mutable struct SingleNode{T} <: MyLinkedListNode{T}
    data::T
    next::Union{SingleNode{T}, Nothing}
end

    \end{minted}
    \caption{Node example for singly linked list in Julia}
    \label{code:node}
    \end{figure}
    Since the field \mintinline{julia}{next} only sometimes contain a
    reference to another node, the data type \mintinline{julia}{Union} was used for field
    declaration -- a union that includes the composite
    data type (for the node in question) and `nothing`.
    
    The first node is conventionally called \emph{the head}, as in the head of the list. 
    The linked list therefore usually only contains one item, the reference to the first node, the 
    head that is, as seen in figure \ref{code:SingleList}. In the authors' implementation, 
    the number of elements that the list holds is also included. 
    The \mintinline{julia}{head}, can
    sometimes contain a reference to nothing, if the list is empty. 
    \begin{figure}[H]
        \centering
    \begin{minted}[
        label= Linked List Datatypes (Custom Struct), 
        linenos, 
        % breaklines, 
        frame = single, 
        fontsize=\footnotesize]{julia}
mutable struct SinglyLinkedList{T} <: MyBasicLinkedList{T}
    head::Union{Nothing, SingleNode{T}}
    n::Int
end
    \end{minted}
    \begin{minted}[
        label= Outer constructs,
        linenos, 
        % breaklines, 
        frame = single, 
        fontsize=\footnotesize]{julia}
# Outer constructs
SinglyLinkedList{T}() where {T} = SinglyLinkedList{T}(nothing, 0)
    \end{minted}
    \caption{Single Linked List in Julia}
    \label{code:SingleList}
    \end{figure}

    The last node is conventionally called \emph{the tail} and, as mentioned earlier, 
    points to ''nothing''. In some implementations, a reference to the tail is also included. 
    However, it was omitted in this study.
    Locating the lists' tail therefore requires one to traverse through the entire list.
    An example on how this can be performed is shown in figure \ref{code:findtail}. 
    \begin{figure}[H]
    \centering
    \begin{minted}[
        label= Locate tail, 
        linenos, 
        % breaklines, 
        frame = single, 
        fontsize=\footnotesize]{julia}
function findtail(ll::MyBasicLinkedList{T}) where {T}
    ll.n == 0  &&  throw(ArgumentError("List is empty"))
    node = ll.head
    while node.next !== nothing
        node = node.next
    end
    return node
end
    \end{minted}
    \begin{minted}[
        label= Locate previous node, 
        linenos, 
        % breaklines, 
        frame = single, 
        fontsize=\footnotesize]{julia}
function findnode_withnext(
    ll::MyAbstractLinkedList{T}, 
    nextnode::MyLinkedListNode{T}
    ) where {T}
    # short-curcuit (error and return) conditions
    isempty(ll) && throw(BoundsError()) 
    nextnode == ll.head && return nothing
    # find next
    node = ll.head 
    while node.next !== nextnode
        (node.next isa Nothing ? 
          (return nothing) : (node = node.next))
    end
    return node
end
    \end{minted}
    \caption{How to locate the tail or previous node of a Linked List, 
    if the reference is unknown}
    \label{code:findtail}
    \end{figure}

    The function \mintinline{julia}{findtail} is used as a step in
    adding/removing items at the end of the list 
    while \mintinline{julia}{findnode_withnext} is when a reference to is unknown.
    For example when removing nodes that are uni-linked, as is the case in 
    the remove method for singly linked list (see \autoref{code:remove!}). 
    Adding items to the end of a collection 
    is conventionally called \mintinline{julia}{push!} while removing is called
    \mintinline{julia}{pop!}. 
    \clearpage 
    
    The function \mintinline{julia}{findtail} 
    is important not only when adding an item, but also when appending another list, to a list. 

    Pushing an item requires it to first be stored in a node, that then gets pointed to 
    by the tail of the list, which has to be found by traversing the entire list. 
    The pushed node effectively becomes the new tail. Appending another list to a list
    requires a similar procedure. As one can read from the code in 
    figure \ref{code:appendfunctions}, the head of the appended list gets attached to the tail 
    of the prepended list.
    
    \begin{figure}[H]
    \begin{minted}[
        % label=codeexample, 
        linenos, 
        % breaklines, 
        % frame = single, 
        fontsize=\footnotesize]{julia}
function push!(sll::SinglyLinkedList{T}, item::T) where {T}
    if isempty(sll)
        sll.head = SingleNode{T}(item, nothing)
    else
        oldtail = findtail(sll) 
        oldtail.next = SingleNode{T}(item, nothing)
    end
    sll.n += 1
    return sll # alt. skriv utan retur
end
function append!(firstlist::SinglyLinkedList, secondlist::SinglyLinkedList)
    if isempty(firstlist)
        firstlist.head = secondlist.head
    else
        oldtail = findtail(firstlist)
        oldtail.next = secondlist.head
    end
    firstlist.n += secondlist.n
    firstlist
end
    \end{minted}
    \caption{The push and append function in Julia}
    \label{code:appendfunctions}
    \end{figure}

    \clearpage

    \section*{Results}
    The final result of the benchmarks are illustrated in the figures below.
    Both diagrammatically visualize the performance of an appending operation. 
    
    \begin{figure}[h]
        \centering
        \includegraphics[width=\textwidth]{./input/fig1onlylistsmicro.pdf}
        \caption{Appending lists}
        \label{fig:appendlists}
    \end{figure}

    \begin{figure}[h]
        \centering
        \includegraphics[width=\textwidth]{./input/fig2withvectorsmilli.pdf}
        \caption{Appending lists compared to vectors}
        \label{fig:appendlistscomparedtovectors}
    \end{figure}

    In figure \ref{fig:appendlists} one can read that the size of the appended list
    does not affect the overall performance. It is the size of the prepended list that is critical. 
    
    But they vastly outperform the same operation with vectors. 
    As seen in the second figure \ref{fig:appendlistscomparedtovectors}
    compares the same operation with vectors. 

    \clearpage
    \section*{Discussion}

Elements of linked lists cannot be as easily (nor as timely) accessed as with arrays. 
It requires one to traverse the list. 
Accessing the last index in an array is done in constant time, $O(1)$, 
while it would require one to traverse the entire linked list -- until one
reaches the node that is linked t ''nothing''. 
The time complexity is therefore linear $O(n)$.
%**{::comment} 
%Ska jag skriva något om nothing, jag skulle kunna ha en fotnot och/eller länka till den delen som beskriver processen med text och kod.
% {:/comment}**
As a consequence *binary search* is not possible using linked lists, 
nor is it possible to locate the maximum element as fast as with vectors 
-- unless, perhaps, it's a linked list that is sorted in descending order.  

On the other hand, TODO ******************* write about space complexity for linked lists!*****
This is due to the fact that vectors size and layout in memory are fixed at the 
time of creation. 
Appending a vector to another requires one to allocate memory for a new larger vector 
that can fit them both. 
All the elements from both vectors need to be copied to the new vector. 

Compared to vectors, it's also easier to insert into or remove items from 
a sequence with linked lists. 
Removing the last node however, requires one to traverse the entire list. 
Time proportional to the length of the list.


% ** skriv om nedan** TODO
%  However, one may need to combine two lists into one. This is the concatenation
%  operation. With chains, the best approach is to attach the second list to the
%  end of the first one. In our implementation of the linked list, this would
%  require one to traverse the first list until the last node is encountered and
%  then set its next pointer to point to the first element of the second list.
%  This requires time proportional to the size of the first linked list. This can
%  be improved by maintaining a pointer to the last node in the linked list.


\end{document}