   
%! Linked Lists
%! Author = Vincent Ferrigan <ferrigan@kth.se>
%! Date = 2022-09-21


% Preamble
\documentclass[a4paper, 11pt]{article}
% Packages
\usepackage[T1]{fontenc}
\usepackage[utf8]{inputenc}
\usepackage[english]{babel}
\usepackage{fontspec}
\usepackage{polyglossia}
\usepackage{microtype}
\setmonofont{DejaVu Sans Mono}[Scale=MatchLowercase]
\usepackage{minted}
\usepackage{latexsym,exscale,stmaryrd,amsmath,amssymb}
\newtheorem{definition}{Definition}
\usepackage{unicode-math}
\usepackage{lmodern}
\usepackage{enumitem}
\usepackage{subcaption}
\usepackage{graphicx}
\usepackage{hyperref}
% \usepackage{natbib}
% \usepackage[nottoc]{tocbibind}
% \usepackage{listings}
\usepackage{xcolor}
\usepackage{siunitx}
\usepackage{tikz}
\usepackage[font=small,labelfont=bf]{caption}
% Addiding JuliaMono
\newfontfamily \JuliaMono {JuliaMono-Regular.ttf}[
    Path      = ./,
    Extension = .ttf
    ]
\newfontface \JuliaMonoMedium{JuliaMono-Regular}
\setmonofont{JuliaMonoMedium}[
    Contextuals=Alternate
]


\title{Linked lists\\ \small{ID1021 Algorithms and Data structures}} %%TODO vilken rubricering
\author{Vincent Ferrigan}

\date{\today}

\begin{document}
    \maketitle
    \section*{Introduction}
    In this study, the performance and comparison of two data structures,
    \emph{Linked Lists} and \emph{Arrays}, were studied. 
    
    This was done through benchmarking -- i.e. comparing the performance of
    the two data structures in terms of managing data. 
    By managing data we mean:
    
    \begin{enumerate}[label*=\arabic*.]
        \item The size of data
        \begin{enumerate}[label*=\arabic*.]
            \item How well do they scale?
            \item Are they dynamic or static in nature? 
        \end{enumerate}
        \item The operations on data
        \begin{enumerate}[label*=\arabic*.]
            \item Appending one to another e.g. appending one array to another
            vis-à-vis appending one linked list to another. 
            \begin{enumerate}[label*=\arabic*.]
                \item Does the type of data structure matter?
                \item Which of the lists affect the overall performance?
                Is the size of the prepending list critical or does the size of
                the appending list matter?
            \end{enumerate}
            \item Deletion/insertion. Does the position matter? That is, is
            there a difference between inserting/deleting an item in the
            beginning or end? Does the type of data structure matter?
            \item Searching for an item.
            \item Updating an item. 
        \end{enumerate}
        \item The performance of certain tasks. 
        Stack implementation is a good example. 
        Do list-based stack perform better than array-based stacks?
    \end{enumerate}
    Not all of above-mentioned will be benchmarked, however, the authors intent is to
    reach certain conclusions on the queries that were stated above. 
    This can be done through studying their implementation. 
            
    \section*{Methods}
    The data structures and stack were implemented in \emph{Julia}. 
    The code was written in \emph{VSCode} and run on \textbf{Julia 1.8.0}. 
    Quick-fixes was done in \emph{Vim}. 
    Some scripts were executed from the \emph{REPL terminal} while others (e.g.
    when using DataFrames and producing Plots) were executed from the
    \emph{Pluto Notebook}. Pluto was also used to stack operations.  

    Julia has a just-in-time (JIT) compilation -- which means that the code is
    dynamically compiled during program run time.     
    It takes time for the JIT compiler to 
    initially load the code and compile it. Therefore, in order not to skew the
    results, \emph{warmup calls} were performed on certain parts of the code
    before they were benchmarked. This in order to avoid including 
    compilation time. The warmup calls were done with the @time macro prior to
    benchmarking.

    % \section*{Learning objectives}
\clearpage

    \section*{The Data Structures}
    This section briefly describes the key differences between arrays and lists 
    
    It is important for the reader to note that in Julia, arrays are 1-based, 
    i.e. started from 1 to N, and that one dimensional arrays are called
    vectors. For consistency, the author has chosen to refer to one dimensional
    arrays as vectors.

    \subsection*{Linked Lists}
    Linked lists are dynamic data structures that unlike vectors, do not require
    to be stored contiguously in memory. They are composed of a chain of 
    nodes that are linked together by pointers (or references). 
    Each node contains at least two parts; an item of any type and 
    the memory address of the next node in the list. 
    The last node in the list contains a null pointer 
    -- i.e. it points to NIL or nothing. 
    Figure \ref{code:SingleNode} shows how a basic node can be implemented in Julia. 
    Instead of pointing to Julia's version of ''nothing'', 
    the author chose to point to a custom type of own creation. 
    As seen in row 3, the type, called \emph{EmptySingleNode}, is immutable 
    and contains nothing. The two constructors in row 9-15 are implemented for 
    convenience. 

    \begin{figure}[H]
    \begin{minted}[
        % label=codeexample, 
        linenos, 
        % breaklines, 
        % frame = single, 
        fontsize=\footnotesize]{julia}
abstract type AbstractSingleNode end

struct EmptySingleNode <: AbstractSingleNode end

mutable struct SingleNode <: AbstractSingleNode
    item::Any
    next::AbstractSingleNode

    function SingleNode(item) 
        new(item, EmptySingleNode())
    end

    function SingleNode(item, next::SingleNode)
        new(item, next)
    end
end
    \end{minted}
    \caption{A basic single node in Julia}
    \label{code:SingleNode}
    \end{figure}
    
    \clearpage
    
    The first node is conventionally called the \emph{head}, as in the head of the list. 
    The linked list therefore only contains one item, the pointer (or reference) to the first node, the 
    head that is, as seen in figure \ref{code:SingleList}
    
    \begin{figure}[H]
    \begin{minted}[
        % label=codeexample, 
        linenos, 
        % breaklines, 
        % frame = single, 
        fontsize=\footnotesize]{julia}
mutable struct SingleLinkedList

    head::AbstractSingleNode

    function SingleLinkedList()
        new(EmptySingleNode())
    end
    function SingleLinkedList(item)
        new(SingleNode(item))
    end
end
        
    \end{minted}
    \caption{Single Linked List in Julia}
    \label{code:SingleList}
    \end{figure}

    The last node is conventionally called the \emph{tail} and, as mentioned earlier, points to ''nothing''.
    Locating the lists tail requires one to traverse through the entire list.
    An example how this can be done is shown in figure \ref{code:findtail}
    
    \begin{figure}[H]
    \begin{minted}[
        % label=codeexample, 
        linenos, 
        % breaklines, 
        % frame = single, 
        fontsize=\footnotesize]{julia}
function findtail(list::SingleLinkedList)
    # short-curcuit condition
    isempty(list) && throw(BoundsError())
    node = list.head
    while !(node.next isa EmptySingleNode)
        node = node.next
    end
    return node
end
    \end{minted}
    \caption{How to loacte the tail of a Linked List in Julia}
    \label{code:findtail}
    \end{figure}
     
    This function is important when appending an item, or another list, to a list. 
    Appending an item requires it to first be stored in a node, that then gets pointed to 
    by the tail of the list, which has to be found by traversing the entire list. 
    The appended node effectively becomes the new tail. Appending another list to a list
    requires a similar procedure. As one can read from the code in 
    figure \ref{code:appendfunctions}, the head of the appended list gets attached to the tail 
    of the prepended list.
    
    \begin{figure}[H]
    \begin{minted}[
        % label=codeexample, 
        linenos, 
        % breaklines, 
        % frame = single, 
        fontsize=\footnotesize]{julia}
function append!(list::SingleLinkedList, item)
    node = SingleNode(item)
    if isempty(list)
        list.head = node
    else
        tail = findtail(list) 
        tail.next = node
    end
end

function append!(firstlist::SingleLinkedList, secondlist::SingleLinkedList)
    if isempty(firstlist)
        firstlist.head = secondlist.head
    else
        tail = findtail(firstlist)
        tail.next = secondlist.head
    end
end
    \end{minted}
    \caption{Append functions in Julia}
    \label{code:appendfunctions}
    \end{figure}

    \clearpage

    \section*{Results}
    The final result of the benchmarks are illustrated in the figures below.
    Both diagrammatically visualize the performance of an appending operation. 
    
    \begin{figure}[h]
        \centering
        \includegraphics[width=\textwidth]{./input/fig1onlylistsmicro.pdf}
        \caption{Appending lists}
        \label{fig:appendlists}
    \end{figure}

    \begin{figure}[h]
        \centering
        \includegraphics[width=\textwidth]{./input/fig2withvectorsmilli.pdf}
        \caption{Appending lists compared to vectors}
        \label{fig:appendlistscomparedtovectors}
    \end{figure}

    In figure \ref{fig:appendlists} one can read that the size of the appended list
    does not affect the overall performance. It is the size of the prepended list that is critical. 
    
    But they vastly outperform the same operation with vectors. 
    As seen in the second figure \ref{fig:appendlistscomparedtovectors}
    compares the same operation with vectors. 

    \clearpage
    \section*{Discussion}
Elements of linked lists cannot be as easily (nor as timely) accessed as with arrays. 
It requires one to traverse the list. 
Accessing the last index in an array is done in constant time, 
while it would require one to traverse the entire linked list -- until one
reaches the node that is linked to the node that contains ''nothing''.
%**{::comment} 
%Ska jag skriva något om nothing, jag skulle kunna ha en fotnot och/eller länka till den delen som beskriver processen med text och kod.
% {:/comment}**
As a consequence *binary search* is not be possible using linked lists, 
since the elements can not be accessed in constant time. 
Nor is it possible to locate the maximum element as fast as with vectors 
-- unless, perhaps, it's a linked list that is sorted in descending order.  
On the other hand, appending lists is more timely than with vectors. 
This is due to the fact that vectors size and layout in memory are fixed at the 
time of creation. 
Appending a vector to another requires one to allocate memory for a new larger vector 
that can fit them both. 
All the elements from both vectors need to be copied to the new vector. 

Compared to vectors, it's also easier to insert into or remove items from 
a sequence with linked lists. 
Removing the last node however, requires one to traverse the entire list. 
Time proportional to the length of the list.


% ** skriv om nedan**
%  However, one may need to combine two lists into one. This is the concatenation
%  operation. With chains, the best approach is to attach the second list to the
%  end of the first one. In our implementation of the linked list, this would
%  require one to traverse the first list until the last node is encountered and
%  then set its next pointer to point to the first element of the second list.
%  This requires time proportional to the size of the first linked list. This can
%  be improved by maintaining a pointer to the last node in the linked list.


\end{document}