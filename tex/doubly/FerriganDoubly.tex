   
%! Linked Lists
%! Author = Vincent Ferrigan <ferrigan@kth.se>
%! Date = 2022-09-21


% Preamble
\documentclass[a4paper, 11pt]{article}
% Packages
\usepackage[T1]{fontenc}
\usepackage[utf8]{inputenc}
\usepackage[english]{babel}
\usepackage{fontspec}
\usepackage{polyglossia}
\usepackage{microtype}
\setmonofont{DejaVu Sans Mono}[Scale=MatchLowercase]
\usepackage{listings}
\usepackage{minted}
\usepackage{latexsym,exscale,stmaryrd,amsmath,amssymb}
\newtheorem{definition}{Definition}
\usepackage{unicode-math}
\usepackage{lmodern}
\usepackage{enumitem}
\usepackage{subcaption}
\usepackage{graphicx}
\usepackage{hyperref}
\usepackage{multirow}

%% Om jag vill referera till ett kod verb av något slag, som void null Int etc
% \usepackage{tcolorbox}
% \newtcbox{\somestuffstyle}{on line,boxrule=0pt,boxsep=0pt,colback=lightgray,top=1pt,bottom=1pt,left=1pt,right=1pt,arc=0pt,fontupper=\ttfamily}

% \usepackage{natbib}
% \usepackage[nottoc]{tocbibind}
\usepackage{xcolor}
\usepackage{siunitx}
\usepackage{tikz}
\usepackage[font=small,labelfont=bf]{caption}
% Addiding JuliaMono
\newfontfamily \JuliaMono {JuliaMono-Regular.ttf}[
    Path      = ./,
    Extension = .ttf
    ]
\newfontface \JuliaMonoMedium{JuliaMono-Regular}
\setmonofont{JuliaMonoMedium}[
    Contextuals=Alternate
]

\title{Doubly linked lists\\ \small{ID1021 Algorithms and Data structures}} %%TODO vilken rubricering
\author{Vincent Ferrigan}

\date{\today}

\begin{document}
    \maketitle
    \section*{Introduction}
    In this study, the performance and comparison of two data structures,
    \emph{Doubly Linked Lists} and \emph{Singly Linked Lists}, were studied. 
    
    This was done through benchmarking -- i.e. comparing the performance of
    the two data structures in terms of managing data. 
    By managing data, we mean:
    
    \begin{enumerate}[label*=\arabic*.]
        \item Size of data. How well do they scale? 
        \item List-operations on data. 
        \begin{enumerate}[label*=\arabic*.]
                \item Does the position matter? 
                That is, is there a difference between inserting/deleting an 
                item in the beginning, middle or end? 
                \item Does the type of data structure matter and why?
            \end{enumerate}
    \end{enumerate}

    The authors intent is to reach certain conclusions on the queries that were stated above. 
    This can also be done through studying their implementation.
    
            
    \section*{Methods}
    All the Data Structures and Algorithms were implemented in \emph{Julia}.
    The Code was mostly written in \emph{VSCode} and run on \emph{Julia 1.8.0}.
    Quick-fixes and editing was, however, done in \emph{Vim}

    Some scripts were executed from the \emph{REPL terminal},  while others (e.g.
    when using data frames, performing benchmarks and producing plots) were executed from the
    \emph{Jupyter Notebook}. 
    
    \subsection*{Tools and packages}
    All tests were performed with the built-in package \emph{Test} and iterative development
    was made possible through \emph{Revise.jl} -- the latter operates by continuously
    scanning the source code for changes, even changes in function defined in
    other modules (including in different files).
    
    The benchmark data was constructed, manipulated and visualized through
    \emph{DataFrames.jl} and \emph{Plots.jl}, while 
    readable formatting was produced through \emph{Formatting.jl} and \emph{Unitful.jl}. 

    \subsection*{The JIT}
    Julia has a just-in-time (JIT) compilation -- which means that the code is
    dynamically compiled during program run time.     
    It takes time for the JIT compiler to 
    initially load the code and compile it. Therefore, in order not to skew the
    results, \emph{warmup calls} were performed on certain parts of the code
    before they were benchmarked. This to avoid including 
    compilation time. The warmup calls were done with the @timed macro prior to
    benchmarking.

    \section*{The Data Structures}
    This section briefly describes the key differences between the two data
    structures. 
    
    \subsection*{Conventions}
    It is important for the reader to note that Julia uses a one-based-numbering
    convention, (i.e. array indices start from 1 to N) and that one dimensional
    arrays are called vectors. 
    For consistency, the author has chosen to refer to one dimensional
    arrays as vectors and apply the 1-based convention when numbering
    sequences of elements more broadly. 

    Function names that ends with a bang (!) mutates the arguments it receives. 
    This name suffix is by convention in Julia. 
    
    Unlike other languages, Julia objects cannot be ''null'' by default. The
    equivalent of \lstinline[language=Python]{None} in Python 
    or \lstinline[language=C]{NULL} and \lstinline[language=C]{void} in C is
    \mintinline{julia}{Nothing}. The Julia convention is to return the value \mintinline{julia}{nothing}, which is a
    singleton instance of type \mintinline{julia}{Nothing}, when such a side effect is
    desired. 

    \subsection*{Linked Lists}
    Linked lists are dynamic data structures that, unlike vectors, do not require
    to be stored contiguously in memory. They are instead composed of a chain of 
    nodes that are linked together by pointers or references. 
    
    The implementation of a linked list requires two data structures;
    one for the actual node and another for the linked-list, the latter containing a
    reference that points to the first node.
    Each node contains at least two parts; an item of any type and 
    the memory address of the next node in the list. 
    The last node in the list contains a next filed that holds no value.  
    -- i.e. it points to \mintinline{julia}{nothing} which, as previously mentioned, is a 
    singleton instance of type ''Nothing''. 
    The nodes for doubly linked, unlike singly linked lists
    also contain a reference that points to the previous node in the list.
    Figure \ref{code:node} shows how two basic nodes can be implemented in Julia. 
    \begin{figure}[H]
        \centering
    \begin{minted}[
        label= Node Data Types (Custom Structs), 
        linenos, 
        % breaklines, 
        frame = single, 
        fontsize=\footnotesize]{julia}
mutable struct SingleNode{T} <: MyLinkedListNode{T}
    data::T
    next::Union{SingleNode{T}, Nothing}
end

mutable struct DoubleNode{T} <: MyLinkedListNode{T}
    data::T
    previous::Union{DoubleNode{T}, Nothing}
    next::Union{DoubleNode{T}, Nothing}
end
    \end{minted}
    \caption{Node examples for singly and doubly linked lists in Julia}
    \label{code:node}
    \end{figure}
    Since the fields \mintinline{julia}{next} and \mintinline{julia}{previous} only sometimes contain a
    reference to another node, the data type \mintinline{julia}{Union} was used for field
    declaration -- a union that includes the composite
    data type (for the node in question) and `nothing`.
    
    The first node is conventionally called \emph{the head}, as in the head of the list. 
    The linked list therefore usually only contains one item, the reference to the first node, the 
    head that is, as seen in figure \ref{code:SingleList}. In the authors' implementation, 
    the number of elements that the lists holds is also included. 
    The \mintinline{julia}{head}, can
    sometimes contain a reference to nothing, if the list is empty. 
    \begin{figure}[H]
        \centering
    \begin{minted}[
        label= Linked List Datatypes (Custom Struct), 
        linenos, 
        % breaklines, 
        frame = single, 
        fontsize=\footnotesize]{julia}
mutable struct SinglyLinkedList{T} <: MyBasicLinkedList{T}
    head::Union{Nothing, SingleNode{T}}
    n::Int
end

mutable struct DoublyLinkedList{T} <: MyBasicLinkedList{T}
    head::Union{Nothing, DoubleNode{T}}
    n::Int
end
    \end{minted}
    \begin{minted}[
        label= Outer constructs,
        linenos, 
        % breaklines, 
        frame = single, 
        fontsize=\footnotesize]{julia}
# Outer constructs
SinglyLinkedList{T}() where {T} = SinglyLinkedList{T}(nothing, 0)
DoublyLinkedList{T}() where {T} = DoublyLinkedList{T}(nothing, 0)
    \end{minted}
    \caption{Single Linked List in Julia}
    \label{code:SingleList}
    \end{figure}
    The last node is conventionally called \emph{the tail} and, as mentioned earlier, 
    points to ''nothing''. In some implementations, a reference to the tail is also included. 
    However, it was omitted in this study.
    Locating the lists' tail therefore requires one to traverse through the entire list.
    An example on how this can be performed is shown in figure \ref{code:findtail}. 
    The function is the same regardless if the list is singly or doubly linked.
    \begin{figure}[H]
    \centering
    \begin{minted}[
        label= Locate tail, 
        linenos, 
        % breaklines, 
        frame = single, 
        fontsize=\footnotesize]{julia}
function findtail(ll::MyBasicLinkedList{T}) where {T}
    ll.n == 0  &&  throw(ArgumentError("List is empty"))
    node = ll.head
    while node.next !== nothing
        node = node.next
    end
    return node
end
    \end{minted}
    \begin{minted}[
        label= Locate previous node, 
        linenos, 
        % breaklines, 
        frame = single, 
        fontsize=\footnotesize]{julia}
function findnode_withnext(
    ll::MyAbstractLinkedList{T}, 
    nextnode::MyLinkedListNode{T}
    ) where {T}
    # short-curcuit (error and return) conditions
    isempty(ll) && throw(BoundsError()) 
    nextnode == ll.head && return nothing
    # find next
    node = ll.head 
    while node.next !== nextnode
        (node.next isa Nothing ? 
          (return nothing) : (node = node.next))
    end
    return node
end
    \end{minted}
    \caption{How to locate the tail or previous node of a Linked List, 
    if the reference is unknown}
    \label{code:findtail}
    \end{figure}
    The function \mintinline{julia}{findtail} is used as a step in
    adding/removing items at the end of the list 
    while \mintinline{julia}{findnode_withnext} is when a reference to is unknown.
    For example when removing nodes that are uni-linked, as is the case in 
    the remove method for singly linked list (see \autoref{code:remove!}). 
    Adding items to the end of a collection 
    is conventionally called \mintinline{julia}{push!} while removing is called
    \mintinline{julia}{pop!}. 
    \clearpage 
    \subsection*{The functions to be benchmarked}
    The study was to benchmark time for $k$ amount of delete and insert operations performed on 
    the two data structures of a growing size of $n$. 
    
    The delete-positions were randomly selected while all insertions were done
    in front of the list. These latter operation is conventionally
    called \mintinline{julia}{pushfirst!} in Julia while the former
    is named \mintinline{julia}{remove!} (in accordance with the assigned task).
    Both functions have two methods each. 
    The method is selected depending on the data type it receives, 
    that is if the data type is a vector or a linked list 
    This feature in Julia is known as \emph{multiple dispatch}, 
    which is similar but not equal to \emph{function overloading} in Java and C++. 
    \begin{figure}[H]
    \centering
    \begin{minted}[
        label=pushfirst! Singly Linked, 
        linenos, 
        % breaklines, 
        frame = single, 
        fontsize=\footnotesize]{julia}
function pushfirst!(
    sll::SinglyLinkedList{T}, 
    node::SingleNode{T}
    ) where {T} 

    if isempty(sll)
        sll.head = node
    else
        oldhead = sll.head
        node.next = oldhead
        sll.head = node
    end
    sll.n += 1
    return sll
end
    \end{minted}
    \begin{minted}[
        label=pushfirst! Doubly Linked,
        linenos, 
        % breaklines, 
        frame = single, 
        fontsize=\footnotesize]{julia}
function pushfirst!(
    dll::DoublyLinkedList{T}, 
    node::DoubleNode{T}
    ) where {T}
    node.previous = nothing # just in case
    if isempty(dll)
        dll.head = node
    else
        oldhead = dll.head
        node.next = oldhead
        oldhead.previous = node
        dll.head = node
    end
    dll.n += 1
    return dll
end
    \end{minted}
    \caption{The pushfirst!()-functions insert items at the beginning of a list. The
    examples are written in Julia and illustrate the key difference between singly
    and doubly linked lists}
    \label{code:pushfirst!}
    \end{figure}
    As one can gather from the implementation of both the pushfirst!-methods in
    figure \ref{code:pushfirst!} and remove!-methods in figure
    \ref{code:remove!}, some additional steps were required for doubly linked lists. 
    This is due to
    the fact that the reference to the previous field must also be updated.
    However, as the benchmarking results will show, being bidirectional has its advances.  
    
    The \mintinline{julia}{remove!} function is called (i.e. used) by several
    functions in the study's implementation, 
    such as the ''pop-functions'' \mintinline{julia}{pop!}, \mintinline{julia}{popat!} and
    \mintinline{julia}{popfirst!}.
    
    To delete a node, one has to connect the previous and the next node together. 
    In order to do so, the references to these nodes need to be known. 
    The node that is up for deletion holds the reference that points to the next. 
    Both methods copy that reference. 
    However, for the singly linked list, the reference to the previous node is unknown.
    Which requires traversing the list, from head until it reaches
    the node that points to the node that is to be deleted. As can be seen when
    comparing row 5 in figure \ref{code:remove!}, the method for the
    singly linked list has to call a separate function that does so 
    and returns the reference to the previous one. 
    The nodes that are manipulated in the method for doubly linked lists, 
    holds two links in both direction. Requiring the action of updating 
    the reference to the previous nodes. 
    If necessary, both methods have to update the 
    lists reference to its head.
    The final step is to replace all links that are held by the removed node
    with \mintinline{julia}{nothing}.
    \begin{figure}[H]
    \centering
    \begin{minted}[
        label = remove! Singly Linked Nodes, 
        linenos, 
        % breaklines, 
        frame = single, 
        fontsize=\footnotesize]{julia}
function remove!(
    sll::SinglyLinkedList{T}, 
    node::SingleNode{T}
    ) where {T}
    previousnode = findnode_withnext(sll, node)
    nextnode = node.next
    (previousnode isa Nothing  # link prev with next
    || (previousnode.next = nextnode))
    (sll.head == node          # update head if nec
      && (sll.head = nextnode))
    node.next = nothing        # nullify removed node
    sll.n -=1                  # update list count
end
    \end{minted}
    \begin{minted}[
        label = remove! Doubly Linked Nodes, 
        linenos, 
        % breaklines, 
        frame = single, 
        fontsize=\footnotesize]{julia}
function remove!(
    dll::DoublyLinkedList{T}, 
    node::DoubleNode{T}
    ) where {T}
    previousnode = node.previous # could be nothing
    nextnode = node.next         # could be nothing
    (nextnode isa Nothing        # link next with prev
      || (nextnode.previous = previousnode))
    (previousnode isa Nothing    # link prev with next
      || (previousnode.next = nextnode))
    (dll.head == node            # update head if nec
      && (dll.head = nextnode))
    node.next = nothing          # nullify removed node
    node.previous = nothing
    dll.n -=1                    # update list count
end
    \end{minted}
    \caption{The remove!() methods deletes a node. 
    The examples are written in Julia and illustrate the 
    key difference between removing a node from a 
    singly linked and doubly linked list}
    \label{code:remove!}
    \end{figure}
    
    \clearpage
    \section*{Results}
    The final result of the benchmarks are illustrated in the figures below.
    Both diagrammatically visualize the performance of editing operation. 
    
    \begin{figure}[h]
        \centering
        \includegraphics[width=0.75\textwidth]{./input/NEWdoubly_fig2v1.pdf}
        \caption{How editing operations for both singly and doubly linked list scale over time
        with the growth of list elements.}
        \label{fig:fig2}
    \end{figure}

    \begin{figure}[h]
        \centering
        \includegraphics[width=0.75\textwidth]{./input/NEWdoubly_fig1v1.pdf}
        \caption{How editing operations on a doubly linked list scale over 
        time with the growth of list elements. One can deduce that the variations are 
        neglectable.}
        \label{fig:fig1}
    \end{figure}

    One can deduce from both the diagrams in \autoref{fig:fig1} and \autoref{fig:fig2} 
    that editing operations on doubly linked lists roughly run in constant time,
    while one can read from \autoref{fig:fig2} that the cost for 
    editing a singly linked list is linear. 
    The variations in \autoref{fig:fig1} are neglectable.
%         \centering
%     \begin{tabular}{|c|c|c|c|c|c|c|}
%         \hline
% \multirow{2}{*}{n} & \multirow{2}{*}{k} & \multicolumn{2}{c|}{Time} & \multirow{2}{*}{Ratio}& \multicolumn{2}{c|}{Growth Ratio}\\
%                                         && SSL & DLL & & SLL & DLL \\

% \hline 
% \SI{100}{k} & \SI{1}{k} & \SI{0.17}{\nano\second} & \SI{0.09}{\nano\second} & \num{1.90} & \num{1.0}  & \num{1.0}\\
% \SI{200}{k} & \SI{1}{k} & \SI{0.36}{\nano\second} & \SI{0.19}{\nano\second} & \num{1.87} & \num{2.2}  & \num{2.2}\\
% \SI{300}{k} & \SI{1}{k} & \SI{0.65}{\nano\second} & \SI{0.30}{\nano\second} & \num{2.16} & \num{3.9}  & \num{3.4}\\
% \SI{400}{k} & \SI{1}{k} & \SI{0.90}{\nano\second} & \SI{0.49}{\nano\second} & \num{1.82} & \num{5.3}  & \num{5.6}\\
% \SI{500}{k} & \SI{1}{k} & \SI{1.28}{\nano\second} & \SI{0.79}{\nano\second} & \num{1.62} & \num{7.6}  & \num{8.9}\\
% \SI{600}{k} & \SI{1}{k} & \SI{1.61}{\nano\second} & \SI{0.82}{\nano\second} & \num{1.97} & \num{9.5}  & \num{9.2}\\
% \SI{700}{k} & \SI{1}{k} & \SI{2.27}{\nano\second} & \SI{0.98}{\nano\second} & \num{2.31} & \num{13.5} & \num{11.1}\\
% \SI{800}{k} & \SI{1}{k} & \SI{2.32}{\nano\second} & \SI{1.36}{\nano\second} & \num{1.71} & \num{13.8} & \num{15.3}\\
% \SI{900}{k} & \SI{1}{k} & \SI{2.63}{\nano\second} & \SI{1.34}{\nano\second} & \num{1.96} & \num{15.6} & \num{15.1}\\
% \SI{1}{m}  & \SI{1}{k} & \SI{3.15}{\nano\second} & \SI{1.54}{\nano\second} & \num{2.05} & \num{18.7} & \num{17.3}\\\hline
% \end{tabular}
%     \caption{How 1 k insert and delete operations scale over time with the
%     growth of list size. SLL stands for Singly Linked List while DLL stands for
%     Doubly linked lists}
%     \label{tab:growth}
%     \end{table}

%     In figure \ref{fig:fig1} one can read that the size of the appended list
%     does not affect the overall performance. It is the size of the prepended list that is critical. 

    \clearpage
    \section*{Discussion}
    % Deleting a node requires one to find the previous node of the node to
    % be deleted and then update the next-field so that it point to correct next
    % node. In order to do so the following pre-step is required -- copying the
    % reference to the next node held by the node that is to be deleted. 
    % Since a singly linked list only permits uni-directional traversal,
    % updating the previous nodes next field
    % the proc
    % requires 

Accessing elements in a doubly linked list (DLL) is more efficient when compared to a
singly linked (SLL). As the benchmark shows, DLL is
has a constant running time while SLL runs in linear timer.
The time complexity of accessing a node 
is constant $O(1)$ if the reference to the node is given, as when editing the
head of the list. 

With DLL, both forward and backward traversal is possible 
while SLL only permits uni-direction traversal. As a consequence, 
the running time for SLL increases if editing is requested between head and tail, 
but not for editing the head. To remove a node from a list, 
one has to connect the previous and the next node together. 
In order to do so, the references to these nodes need to be known. 
The node that is up for removal holds the reference that points to the next. 
Unlike DLL, the reference to the previous node is unknown in a Singly Linked List.
Which requires traversing the list, from head until it reaches the node that 
points to the node that is to be removed. Worst case, it may cost in linear time,
if the node for removal is at the end of the list. In other words, linear time,
that gives editing a SLL a time complexity of $O(1)$ 

Unlike push!, a pushfirst!() does not
require a traversal. Therefore, adding an item to the beginning 
of a list does not have a negative effect on SLL.  

The core issue is that to update a list, one has to obtain the 
reference to the node that is before the location that is to be amended. 
Updating the references themselves is a
constant time operation, so the real issue is how fast the algorithm can 
find that previous node. 

\subsubsection*{Room for improvement}
Adding a field that keeps track of the reference to the tail would enhance the 
performance of the push!(), pop!() and append!() operations on lists, regardless of how
their nodes are linked. All three involving the tail. 
The time complexity would change to constant time
$O(n)$.  Not maintaining a reference to the tail requires a time-cost proportional to the 
size of the list. So why not add an extra field? 

\end{document}