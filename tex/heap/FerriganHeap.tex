
%! FerriganHeap.tex
%! Author = Vincent Ferrigan <ferrigan@kth.se>
%! Date = 2022-11-04


% Preamble
\documentclass[a4paper, 11pt]{article}
% Packages
\usepackage[T1]{fontenc}
% \usepackage[utf8]{inputenc} % ska den tas bort iom lua?
% \usepackage[utf8]{luainputenc}
\usepackage[english]{babel}
\usepackage{fontspec}
\usepackage{microtype}
\setmonofont{DejaVu Sans Mono}[Scale=MatchLowercase]
\usepackage{listings}
\usepackage{minted}
\usepackage{latexsym,exscale,stmaryrd,amsmath,amssymb}
\newtheorem{definition}{Definition}
\usepackage{unicode-math}
\usepackage{lmodern}
\usepackage{enumitem}
\usepackage{subcaption}
\usepackage{graphicx}
\usepackage{hyperref}
\usepackage{multirow}
\usepackage{diagbox} % For diagonal lines in tabular
\usepackage{booktabs}
% \usepackage{paralist}

%% Om jag vill referera till ett kod verb av något slag, som void null Int etc
% \usepackage{tcolorbox}
% \newtcbox{\somestuffstyle}{on line,boxrule=0pt,boxsep=0pt,colback=lightgray,top=1pt,bottom=1pt,left=1pt,right=1pt,arc=0pt,fontupper=\ttfamily}

\usepackage[
    backend=biber,
    % hyperref=true,
    maxnames=3, 
    minnames=1, 
    nohashothers=false
    bibencoding=utf8, % eventuellt
    style=apa,
    % citestyle=apa,
    pluralothers=true,
    natbib=true
    % sorting=nyt
    % autocite=inline
    ]{biblatex}
\DefineBibliographyStrings{english}{andothers={et. al}, and={&}}
\DeclareLanguageMapping{english}{english-apa}
\addbibresource{references.bib} % hör till referenser
% \addbibresource{../references.bib} % hör till referenser
% \usepackage[backend=biber,style=apa,natbib=true,sorting=nyt]{biblatex}
% \addbibresource{references.bib}
% \usepackage{natbib}
\usepackage{csquotes}
\usepackage[nottoc]{tocbibind}
\usepackage{xcolor}
\usepackage{siunitx}
\usepackage{tikz}
\usepackage[font=small,labelfont=bf]{caption}
% Addiding JuliaMono
\newfontfamily \JuliaMono {JuliaMono-Regular.ttf}[
    Path      = ./,
    Extension = .ttf
    ]
\newfontface \JuliaMonoMedium{JuliaMono-Regular}
\setmonofont{JuliaMonoMedium}[
    Contextuals=Alternate
]

\title{A Heap or Priority Queue\\ \small{ID1021 Algorithms and Data structures}} %%TODO VILKEN RUBRICERING
\author{Vincent Ferrigan}

\date{\today}

\begin{document}
    \maketitle
    \section*{Introduction}
    \emph{Priority Queues} are data structures that support two operations; removing items 
    in a specific order and adding item in a way the preserves that order. 
    In other words, 
    the items are perceived to have a certain priority \parencite{Segeqick2011Alg4th}.
    In this study, each item holds an integer as \emph{key}, 
    where small key-numbers have higher priority. 
    The key being a unique integer for a node that also holds a specific data (value)
    
    While \emph{stacks} and \emph{queues} remove items depending on the order in
    which they were added, the 
    so-called \emph{FIFO} and \emph{LIFO approach}, 
    the priority queue in this study 'removes the minimum' - where the 'minimum' is the
    item with the highest priority. 

    The study will compare different implementations of priority queues, 
    where focus will lie on the \emph{binary heap}. It will also look at performance,
    which will be done through benchmarking. 

    \section*{Methods}
    \label{sec:methods}
    All the Data Structures and Algorithms were implemented in \emph{Julia}.
    The Code was mostly written in \emph{VSCode} and run on \emph{Julia 1.8.0}.
    Quick-fixes and editing was, however, done in \emph{Vim}. 
    Some scripts were executed from the \emph{REPL terminal},  while others (e.g.
    when using data frames, performing benchmarks and producing plots) 
    were executed from the \emph{Jupyter Notebook}. 
    % Ska jag lägga till länk till min github? To follow the progress.....
    % men då måste notebooken läggas upp.
    
    \subsection*{Tools and packages}
    All tests were performed with the built-in package \emph{Test} and 
    iterative development was made possible through 
    \emph{Revise.jl} -- the latter operates by continuously
    scanning the source code for changes, even changes in functions defined in
    other modules (including modules written in different files). 
    Version control was done through \emph{Git} and \emph{Git-Hub} and the paper
    was written in \emph{\LaTeX} and compiled with \emph{LuaTeX} and \emph{Biber}.
    
    The benchmark data was constructed, manipulated and visualized through
    \emph{DataFrames.jl} and \emph{Plots.jl}, 
    while readable formatting was produced through 
    \emph{Formatting.jl} and \emph{Unitful.jl}. 

    \subsection*{The JIT}
    Julia has a just-in-time (JIT) compilation -- which means that the code is
    dynamically compiled during program run time.     
    It takes time for the JIT compiler to 
    initially load the code and compile it. Therefore, in order not to skew the
    results, \emph{warm-up calls} were performed on certain parts of the code
    before they were benchmarked. This to avoid including 
    compilation time. The warm-up calls were done with the @timed macro prior to
    benchmarking.

    \section*{The Data Structure/Algorithms} %% TODO CHANGE DEPENDING ON ASSIGNMENT
    xxxxxx
    \subsection*{Conventions}
    xxxxxx

    \begin{figure}[h]
        \centering
        %% you can have several subfigures or minteds
        %% TODO: add label/caption on minted, e.g. Outer Method
    \begin{minted}[
        label= XXXXXXX, 
        linenos, 
        % breaklines, 
        frame = single, 
        fontsize=\footnotesize]{julia}

    \end{minted}
    \caption{xxx} %% TODO ADD CAPTION
    \label{code:xxx} %% TODO ADD LABEL
    \end{figure}

    \section*{Results}
    \label{sec:results}
    xxxxxx

    % \begin{figure}[h] % You can have subfigures !! If necessary 
    %     \centering
    %     \includegraphics[width=0.8\textwidth]{./input/xxxxx.pdf} %% TODO: add pdf
    %     \caption{xxxx}%% TODO: add caption
    %     \label{fig:fig1} %% TODO: add correct fignbr /label
    % \end{figure}

    \begin{table}[h]
        \centering
        \small
\begin{tabular}{|c|c|c|} % TODO: correct columns

\end{tabular}
    \caption{xxxx} % TODO: add caption
    \label{tab:xxx} % TODO: add correct label
    \end{table}
\hline

    \section*{Discussion}
    \label{sec:discussion}
    xxxxxx
    \section*{EXAMPLE SECTION WITH REF}
    xxxxxx 
    \hyperref[sec:results]{Results}
    xxxxxx 
    \autoref{code:xxx}
    % xxxxxx 
    % \autoref{fig:fig1}

    xxxxxx with with apren cite 
    \parencite{Segeqick2011Alg4th}

    xxxxxx with with paren cite 
    \parencite{CormenThomasH2022ItA}

    xxxxxx with with text cite 
    \textcite{Segeqick2011Alg4th}
    xxx

    xxxxxx with with text cite 
    \citep{Segeqick2011Alg4th}
    xxx

    xxxxxx with with text cite 
    \citet{Segeqick2011Alg4th}
    xxx

    xxxxxx with with text cite 
    \citep{CormenThomasH2022ItA}
    xxx


    %\section{Appendix A - Referenser}
\printbibliography
\end{document}