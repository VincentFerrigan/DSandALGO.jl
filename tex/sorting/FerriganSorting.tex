   
%! Author = Vincent Ferrigan <ferrigan@kth.se>
%! Date = 2022-09-02


% Preamble
\documentclass[a4paper, 11pt]{article}
% Packages
\usepackage[T1]{fontenc}
\usepackage[utf8]{inputenc}
\usepackage[english]{babel}
\usepackage{fontspec}
\usepackage{polyglossia}
\usepackage{microtype}
\setmonofont{DejaVu Sans Mono}[Scale=MatchLowercase]
\usepackage{minted}
\usepackage{latexsym,exscale,stmaryrd,amsmath,amssymb}
\newtheorem{definition}{Definition}
\usepackage{unicode-math}
\usepackage{lmodern}
\usepackage{enumitem}
\usepackage{subcaption}
\usepackage{graphicx}
\usepackage{hyperref}
% \usepackage{natbib}
% \usepackage[nottoc]{tocbibind}
% \usepackage{listings}
\usepackage{xcolor}
\usepackage{siunitx}
\usepackage{tikz}
\usepackage[font=small,labelfont=bf]{caption}
% Addiding JuliaMono
\newfontfamily \JuliaMono {JuliaMono-Regular.ttf}[
    Path      = ./,
    Extension = .ttf
    ]
\newfontface \JuliaMonoMedium{JuliaMono-Regular}
\setmonofont{JuliaMonoMedium}[
    Contextuals=Alternate
]


\title{Searching in a sorted array\\ \small{HID1021 Algorithms and data structures}} %%TODO vilken rubricering
\author{Vincent Ferrigan}

\date{\today}

\begin{document}
    \maketitle
    \section*{Introduction}
    In this paper, the performance and implementation of three sorting algorithms,  
    \emph{selection sort}, \emph{insertion sort} and \emph{merge sort}, 
    were analyzed. This was done through benchmarking -- i.e. comparing the performance on
    different sets of data solving different types of problems. 
    For example, one might ask the following two fundamental questions
    
    \begin{itemize}
        \item How well do these algorithms' scale over time, with the growth of input data?
        \item Does the quality of data affect the performance?
        E.g. what if the elements in a list are already in order? 
        What if they are partially in order?
    \end{itemize}

    The author of this study has therefore chosen to compare these algorithms on 
    two types of data; data that needs to be sorted and data that already
    happens to be in order. 
    

    \section*{Methods}
    The algorithms were implemented in \emph{Julia}. The code was written in 
    \emph{VSCode} and run on \textbf{Julia 1.8.0}. Some scripts were executed
    from the \emph{REPL terminal} while others (e.g. when using DataFrames and
    Plots) were executed from the \emph{Jupyter Notebook}. 

    Julia has a just-in-time (JIT) compilation -- which means that the code is
    dynamically compiled during program run time.     
    It takes time for the JIT compiler to 
    initially load the code and compile it. Therefore, in order not to skew the
    results, \emph{warmup calls} were performed on certain parts of the code
    before they were benchmarked. This in order to avoid including 
    compilation time. The warmup calls were done with the @time macro prior to
    benchmarking.
    

\clearpage

    \section*{The Algorithms}
    This section briefly describes the algorithms and exemplifies their 
    implementation with Julia code that was written by the author of this study. 
    
    It is important for the reader to note that in Julia, arrays are 1-based, 
    i.e. started from 1 to N, and that one dimensional arrays
    are called vectors. For consistency, the author has chosen to refer to 
    one dimensional arrays as vectors.

    \subsection*{Selection Sort}
    The simplest of the three sorting algorithms is \emph{Selection Sort}. 
    It performs sorting by first selecting the smallest item in a list
    and swapping it with the first entry. Then it repeats this act in order to
    find the next smallest item to swap with the second entry. 
    It repeatably searches through the remaining items in order to find and
    select the smallest item to move into the next position -- until all
    positions are filled. See figure \ref{code:selectionsort} for how selection sort
    might be implemented in Julia. 

    In row 3 one can find the \emph{short-circuit pre-condition} for this function 
    (no need to sort a vector that consist of less than 2 items).
    The condition stated in row 10 is probably unnecessary, but was added by the author
    to avoid the algorithm's absurdity of swapping the selected item with itself
    if already positioned correctly. Row 11 is an example of how relatively easy it is 
    to swap two values in Julia compared to other programming languages.

    \begin{figure}[H]
    \begin{minted}[
        % label=codeexample, 
        linenos, 
        % breaklines, 
        % frame = single, 
        fontsize=\footnotesize]{julia}
function selectionsort!(v::Vector)
    N = length(v)
    N > 1 || throw(ArgumentError("Vector $v has less than 2 item"))

    for i = 1:(N - 1)
        min_index = i
        for j = (i + 1):N
            if v[j] < v[min_index]  min_index = j end
        end
        if min_index != i  
            v[i], v[min_index] = v[min_index], v[i] 
        end
    end
end
    \end{minted}
    \caption{Selection Sort in Julia}
    \label{code:selectionsort}
    \end{figure}
    
    \clearpage
    \subsection*{Insertion Sort}
    In each iteration of this algorithm, an element is inserted into an 
    already sorted part of the list.  The insertion requires that all items,
    with indices in between the new and previous position of the inserted item, 
    be moved one step. The new position for the inserted item is not its final one.
    It may be required to move to make room for a smaller item.
    This ''move'' is expressed as a while-loop in the Julia 
    implementation example below (see row 7-10 in figure
    \ref{code:insertionsort}). The above steps are 
    iterated until the entire list is in order. 

        
    \begin{figure}[H]
    \begin{minted}[
        % label=codeexample, 
        linenos, 
        % breaklines, 
        % frame = single, 
        fontsize=\footnotesize]{julia}
function insertionsort!(v::Vector)
    N = length(v)
    N > 1 || throw(ArgumentError("Vector $v has less than 2 items"))

    for i = 2:N
        j = i
        while j > 1  &&  v[j] < v[j - 1]
            v[j], v[j - 1] = v[j - 1], v[j]
            j -= 1
        end
    end
end
    \end{minted}
    \caption{Insertion Sort in Julia}
    \label{code:insertionsort}
    \end{figure}
    
    \clearpage
    \subsection*{Merge Sort}
    The main idea behind \emph{Merge Sort} is based on the
    \emph{divide-and-conquer method}.
    That is \emph{dividing} the (sorting) problem into independent subproblems. 
    \emph{Conquer} the subproblems by solving them recursively and later \emph{combine} the 
    subsolutions to solve the overall problem. 
    The algorithm has therefore three parts:
    \begin{enumerate}
        \item Divide the vector into two halves,
        \item conquer each half by sorting them recursively and
        \item combine the results by merging the two halves to obtain one fully
        sorted vector.
    \end{enumerate}
    The algorithm is implemented as a recursive function. 
    Every recursive function starts with a \emph{''do-again-test''},
    which in the implementation example, 
    is written as a \emph{short-circuit return condition} 
    (see row 2 in figure \ref{code:mergesort}).
    The recursion ''bottoms out'' when the subvector to be sorted has just one
    element. It occurs in the implementation example when ''top'' meets ''bottom'' 
    as stated in the ''do-again-test''. 

    \begin{figure}[H]
    \begin{minted}[
        % label=codeexample, 
        linenos, 
        % breaklines, 
        % frame = single, 
        fontsize=\footnotesize]{julia}
function mergesort!(v::Vector, bottom = 1, top = length(v)) 
    top <= bottom  &&  return

    middle = ÷(bottom + top, 2)
    mergesort!(v, bottom, middle)  # Sort bottom half
    mergesort!(v, middle + 1, top) # Sort top half
    merge!(v, bottom, middle, top) # Merge results
end

    \end{minted}
    \caption{Merge Sort in Julia}
    \label{code:mergesort}
    \end{figure}

    \clearpage
    The merge part of the algorithm is written by the author as a \emph{''helper function''}.
    The code (see figure \ref{code:merge}) with its comments explains how the merge-part of
    the algorithm operates. 

    \begin{figure}[H]
    \begin{minted}[fontsize=\footnotesize]{julia}
function merge!(v::Vector, bottom, middle, top)
    i = bottom          # bottom index
    j = middle + 1      # top index

    # Copy vector v to a temporary vector
    tempv = Vector(undef, top)
    for k = bottom:top
         tempv[k] = v[k]
    end

    for k = bottom:top
        # merge
        if i > middle       # when the bottom is done
            v[k] = tempv[j]
            j += 1
        elseif j > top      # when the top is done 
            v[k] = tempv[i]
            i += 1

        # add ''remainders'' according to size 
        elseif tempv[j] < tempv[i] 
            v[k] = tempv[j]
            j += 1
        else
            v[k] = tempv[i]
            i += 1
        end
    end
end
    \end{minted}
    \caption{The helper-function to Merge Sort in Julia}
    \label{code:merge}
    \end{figure}    \clearpage
    \section*{Results}
    The final result of the benchmarks are illustrated diagrammatically 
    in three figures below. \ref{fig:smaldata} and \ref{fig:bigdata} were 
    performed on randomly ordered vectors, while \ref{fig:sorted} was 
    performed on already sorted vectors. 
\begin{figure}[h]
    \centering
    \begin{subfigure}[h]{0.45\textwidth}
        \centering
        \includegraphics[width=\textwidth]{./input/fig1smallsmall.pdf}
        \caption{Smaller range}
        \label{fig:smaldata}
    \end{subfigure}
    \hfill
    \begin{subfigure}[h]{0.45\textwidth}
        \centering
        \includegraphics[width=\textwidth]{./input/fig1big.pdf}
        \caption{Greater range}
        \label{fig:bigdata}
    \end{subfigure}
    % \hfill
    \begin{subfigure}[h]{0.45\textwidth}
        \centering
        \includegraphics[width=\textwidth]{./input/fig1sorted.pdf}
        \caption{Ordered data}
        \label{fig:sorted}
    \end{subfigure}
    \caption{Growth of data scale over time}
    \label{fig:benchmarks}
\end{figure}

    \section*{Discussion}
    Selection Sort passes through a vector without passing info about the vector to
    the next search operation, which means that the quality of the data does not
    affect its performance. It does not matter if the data is already in order, 
    semisorted or even if it consists entirely of entries with equal value, 
    it still takes the same amount of time to sort as it would do with a 
    randomly ordered vector. 
    
    Regardless of how the vector looks, the loop gets 
    executed exactly $N-1$ times. In other words, there are no best or worst cases. 
    The running time is
    dominated by the number of compares which are $\frac{N^2}{2}$. 
    Which gives us a time complexity of $O(n^2)$. The time complexity for
    this method is well illustrated in the benchmarks visually represented as 
    diagrams in the figure \ref{fig:benchmarks}. 

    As one can clearly read from the benchmark results, the performance of 
    Insertion Sort heavily depends on the initial order of the item in the vector.
    As shown in figure \ref{fig:sorted}, where the benchmark was performed on already 
    sorted vectors, Insertion Sort outperformed not only 
    Selection Sort but also Merge Sort. This is due to the fact that Insertion Sort
    ''skips'' the sorted items. The study can therefore conclude that for partially 
    sorted vectors, insertion sort might be quite efficient. 
    
    E.g. it would 
    probably be the algorithm of choice for sorting data where a small vector
    has been appended to a large sorted vector. However, even though it might excel 
    in certain cases, its time complexity is still based on its worst case scenario, 
    which like Selection Sort is $O(n^2)$.

    For greater data sets, that are randomly ordered, 
    Merge Sort is to be preferred -- as the benchmark results 
    in figure \ref{fig:bigdata} shows. 
    It has a time complexity of $O(n\log n)$. 
    Nonetheless, since the algorithm is recursive, one
    can conclude that the memory cost is greater than for the other two algorithms. 
    The merge function in the example code in figure \ref{code:merge}, 
    has to allocate memory for the temporary vector (in which elements get copied to).
    This would require a lot of data space for large vectors. \emph{The question is 
    for how big of a data set would force the programmer to choose a 
    different algorithm all together?}

    Fortunately, there are plenty of sorting algorithms to choose from. \emph{Shell Sort}, 
    \emph{Jump Sort}, \emph{Heap Sort} and \emph{Quick Sort}... to name a few.  


    
\end{document}