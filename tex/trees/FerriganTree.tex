   
%! Linked Lists
%! Author = Vincent Ferrigan <ferrigan@kth.se>
%! Date = 2022-09-21


% Preamble
\documentclass[a4paper, 11pt]{article}
% Packages
\usepackage[T1]{fontenc}
\usepackage[utf8]{inputenc}
\usepackage[english]{babel}
\usepackage{fontspec}
\usepackage{polyglossia}
\usepackage{microtype}
\setmonofont{DejaVu Sans Mono}[Scale=MatchLowercase]
\usepackage{listings}
\usepackage{minted}
\usepackage{latexsym,exscale,stmaryrd,amsmath,amssymb}
\newtheorem{definition}{Definition}
\usepackage{unicode-math}
\usepackage{lmodern}
\usepackage{enumitem}
\usepackage{subcaption}
\usepackage{graphicx}
\usepackage{hyperref}
\usepackage{multirow}
% \usepackage{paralist}
\usepackage{enumitem}

%% Om jag vill referera till ett kod verb av något slag, som void null Int etc
% \usepackage{tcolorbox}
% \newtcbox{\somestuffstyle}{on line,boxrule=0pt,boxsep=0pt,colback=lightgray,top=1pt,bottom=1pt,left=1pt,right=1pt,arc=0pt,fontupper=\ttfamily}

% \usepackage{natbib}
% \usepackage[nottoc]{tocbibind}
\usepackage{xcolor}
\usepackage{siunitx}
\usepackage{tikz}
\usepackage[font=small,labelfont=bf]{caption}
% Addiding JuliaMono
\newfontfamily \JuliaMono {JuliaMono-Regular.ttf}[
    Path      = ./,
    Extension = .ttf
    ]
\newfontface \JuliaMonoMedium{JuliaMono-Regular}
\setmonofont{JuliaMonoMedium}[
    Contextuals=Alternate
]

\title{Trees on stacks and queues\\ \small{ID1021 Algorithms and Data structures}} %%TODO vilken rubricering
\author{Vincent Ferrigan}

\date{\today}

\begin{document}
    \maketitle
    \section*{Introduction}
    In this paper, the performance and implementation of \emph{binary trees} 
    was studied and analyzed. A \emph{lookup algorithm} was also compared with
    the binary search algorithm, which was studied in a previous paper. 
    The was done through benchmarking -- i.e. comparing the performance of
    the two data algorithms in terms of managing data. E.g. how well do these
    algorithms' scale with the size of the input data? 
    Does the shape of the tree matter? That is, when 
    constructing a tree, does the order of adding nodes matter?

    The final task was to implement an iterator for traversing a binary tree 
    based on the \emph{Depth First} approach. Which was implemented using a stack. 
    The question is, in what order? Inorder, preorder or postorder? 
    All these three \emph{Traversal Algorithms} where studied. 
    
    The authors intent is to reach certain conclusions on the queries that were
    stated above. 
    
    \section*{Methods}
    All the Data Structures and Algorithms were implemented in \emph{Julia}.
    The Code was mostly written in \emph{VSCode} and run on \emph{Julia 1.8.0}.
    Quick-fixes and editing was, however, done in \emph{Vim}

    Some scripts were executed from the \emph{REPL terminal},  while others (e.g.
    when using data frames, performing benchmarks and producing plots) were executed from the
    \emph{Jupyter Notebook}. 
    
    \subsection*{Tools and packages}
    All tests were performed with the built-in package \emph{Test} and iterative development
    was made possible through \emph{Revise.jl} -- the latter operates by continuously
    scanning the source code for changes, even changes in function defined in
    other modules (including in different files).
    
    The benchmark data was constructed, manipulated and visualized through
    \emph{DataFrames.jl} and \emph{Plots.jl}, while 
    readable formatting was produced through \emph{Formatting.jl} and \emph{Unitful.jl}. 

    \subsection*{The JIT}
    Julia has a just-in-time (JIT) compilation -- which means that the code is
    dynamically compiled during program run time.     
    It takes time for the JIT compiler to 
    initially load the code and compile it. Therefore, in order not to skew the
    results, \emph{warmup calls} were performed on certain parts of the code
    before they were benchmarked. This to avoid including 
    compilation time. The warmup calls were done with the @timed macro prior to
    benchmarking.

    \section*{The Data Structure}
    This section briefly describes the key differences between \emph{binary trees} 
    and other data structures -- such as \emph{linked lists} and \emph{vectors}.
    For clarity, the reader is given meaning to the terms dealing 
    with binary trees -- not only with descriptions of the
    ''tree-terms'' used in this paper, but also with a specification 
    of the overall implementation in Julia. The reader will also be provided with 
    code examples when necessary.

    \subsection*{Conventions}
    It is important for the reader to note that Julia uses a one-based-numbering
    convention, (i.e. array indices start from 1 to N) and that one dimensional
    arrays are called vectors. 
    For consistency, the author has chosen to refer to one dimensional
    arrays as vectors and apply the 1-based convention when numbering
    sequences of elements more broadly. 
    
    Unlike other languages, Julia objects cannot be ''null'' by default. The
    equivalent of \lstinline[language=Python]{None} in Python or
    \lstinline[language=C]{NULL} and \lstinline[language=C]{void} in C is
    \mintinline{julia}{Nothing}. The Julia convention is to return the value
    \mintinline{julia}{nothing}, which is a singleton instance of type
    \mintinline{julia}{Nothing}, when such a side effect is
    desired. 

    \subsection*{The Binary Tree}
    Like \emph{linked lists}, \emph{binary trees} are dynamic data structures that, unlike \emph{vectors}, 
    do not require to be stored contiguously in memory. They are instead composed of \emph{nodes} 
    that are linked together by pointers or references. 
    Like the implementation of linked lists, the tree requires two data structures; one
    for the actual node and another for the tree, the latter containing a
    reference that points to the first node. 
    Each node in the binary tree that was implemented in this paper, contains five parts;
    % \begin{inparaenum}[i)]
    \begin{enumerate}[itemsep=1pt]
        \item a \mintinline{julia}{key} of any type \mintinline{julia}{K}
        \item a \mintinline{julia}{value} of any type \mintinline{julia}{V} 
        that is associated with the key
        \item a reference (memory address) to the linked node \mintinline{julia}{left}
        \item a reference (memory address) to the linked node \mintinline{julia}{right} and
        \item the number of nodes \mintinline{julia}{n} in the tree/subtree. 
        A number that that includes the node itself as well as all its descendants.
    \end{enumerate}
    % \end{inparaenum}

    The fields that contain references to other nodes may point to nothing 
    which, as previously mentioned, is a singleton instance of type \mintinline{julia}{Nothing}.
    \autoref{code:node} shows how the above-mentioned nodes were implemented in Julia. 
    \begin{figure}[H]
        \centering
    \begin{minted}[
        % label=codeexample, 
        linenos, 
        % breaklines, 
        % frame = single, 
        fontsize=\footnotesize]{julia}
mutable struct BTNode{K, V} <:MyAbstractTreeNode{K, V} 
    key::K
    value::V
    left::Union{BTNode{K, V}, Nothing}
    right::Union{BTNode{K, V}, Nothing}
    n::Int64 
end
    \end{minted}
    \caption{Node example for Binary Search Trees in Julia}
    \label{code:node}
    \end{figure}

    Since the fields \mintinline{julia}{left} and \mintinline{julia}{right} only sometimes 
    contain a reference to another node, the data type \mintinline{julia}{Union} was used for their 
    field declaration -- a union that includes the composite
    data type (for the node in question) and `nothing`.

    The first node in the tree is conventionally called \emph{the root}, as in the root of the tree.
    The tree therefore usually only contains one item, the reference to the first node, the root that is.
    See \autoref{code:binarytree}. Apart from the root, 
    each node in a binary tree is pointed to by just one other node, which is called its \emph{parent}. 
    At most, every node can have two children, the first is called the \emph{left child}
    and the second is called the \emph{right child}.  
    
    \begin{figure}[H]
        \centering
    \begin{minted}[
        % label=codeexample, 
        linenos, 
        % breaklines, 
        % frame = single, 
        fontsize=\footnotesize]{julia}
mutable struct BTree{K, V} <: MyBinaryTree{K, V}
    root::Union{Nothing, BTNode{K, V}}
    # depth::Int
end

# Outer constructs
BTree{K, V}() where {K, V} = BTree{K, V}(nothing)
# BTree{T}() where {T} = BTree{T}(nothing, 0)
    \end{minted}
    \caption{A Binary Search Tree in Julia.
    The commented out code points to the fact that the number of nodes that a tree holds can also be
    included in a BST implementation.}
    \label{code:binarytree}
    \end{figure}

    A node of a tree is called a \emph{leaf} if it has no children. Nodes that
    have at least one child are called \emph{internal nodes}. 
    The root is internal node unless it is the only node in the tree, in which case it is a leaf. 
    The field \mintinline{julia}{root} in the binary tree (as shown in the
    example in \autoref{code:binarytree})
    can sometimes contain a reference to nothing, in which case the tree is empty.
    The tree rooted at the left child
    of a node is called the \emph{left subtree} of this node, and the tree rooted at the
    right child of the node is called the \emph{right subtree} of the node.
    Since each subtree are themselves binary trees. Two nodes with the same 
    parent are called \emph{siblings}. In this study, the nodes holds neither reference to its parent 
    nor to its sibling -- which affects the traversal of the tree 
    (see subsection \hyperref[sec:treetraversal]{Tree Traversal}).

    The binary tree in this paper is ordered as a \emph{Binary Search Tree (BST)}, that is, 
    the key of a node must be unique and larger than the keys of all nodes in
    its left subtree and smaller than the keys of all the nodes in its right
    subtree. The siblings are therefore ordered from left to right.

    The procedure of adding a new item (consisting of a key and its associated value)
    to the BST requires comparing keys of
    nodes that already exist in the BST. Starting at the root and moving left
    if the new items key is smaller than visited node or right if it is larger. 
    The function written for this procedure, as shown in 
    \autoref{code:add!}, adds a new node (leaf) that maps
    given key to value. If the given key is already present, the value that is
    mapped to it gets updated with given value. This function is recursive in nature and 
    is therefore implemented with two methods depending on 
    the data type it receives,  that is if the data type is a tree or a node. 
    This feature in Julia is known as
    \emph{multiple dispatch}, which is similar but not equal to \emph{function
    overloading} in Java and C++. The function name ends with a bang (!) since it
    mutates the arguments it receives. This name suffix is by convention in Julia.

    \begin{figure}[H]
        \centering
    \begin{minted}[
        % label=codeexample, 
        linenos, 
        % breaklines, 
        % frame = single, 
        fontsize=\footnotesize]{julia}
function add!(
    tree::BTree{K, V}, 
    key::K, value::V
    ) where {K,V}
    
    tree.root = add!(tree.root, key, value)
    tree
end

function add!(
    node::Union{BTNode{K, V}, Nothing}, 
    key::K, 
    value::V
    ) where {K,V}

    if isa(node, Nothing)  
        node = BTNode{K, V}(key, value, nothing, nothing, 1)
        return node
    end

    if key < node.key
        node.left = add!(node.left, key, value)
    elseif key > node.key
        node.right= add!(node.right, key, value)
    else
        node.value = value
    end

    node.n = size(node.left) + size(node.right) + 1
    return node
end
    \end{minted}
    \caption{The insert or update function 'add!' in Julia}.
    \label{code:add!}
    \end{figure}

    The procedure of locating an item acts similarly to adding one to
    a BST. 
    The function, which also consist of two methods, is named 'lookup' (in
    accordance with the assigned task) and its Julia code is shown in figure
    \autoref{code:lookup}. The function locates and returns the value associated
    to given key. The methods return 'nothing' if the given key is not found. 
    
    It is this function that is to benchmarked to 
    a binary search algorithm (see \hyperref[sec:results]{Section Results})


    \begin{figure}[H]
        \centering
    \begin{minted}[
        % label=codeexample, 
        linenos, 
        % breaklines, 
        % frame = single, 
        fontsize=\footnotesize]{julia}
lookup(tree::BTree{K,V}, key::K) where {K,V} = lookup(tree.root, key)

function lookup(node::Union{BTNode{K, V}, Nothing}, key::K) where {K,V}
    isa(node, Nothing) && return nothing

    if key < node.key
        lookup(node.left, key)
    elseif key > node.key
        lookup(node.right, key)
    else
        return node.value
    end
end
    \end{minted}
    \caption{The locate function 'lookup' in Julia}
    \label{code:lookup}
    \end{figure}

    \subsection*{The Traversal Algorithms}
    \label{sec:treetraversal}
    
    A tree traversal refers to the process of visiting each node in a tree at least once. This 
    ''visit'' can inquire some kind of action, e.g. accessing each node for printing their 
    keys and their associated values. The question of order is of same
    importance as visiting each node only once. Downward or sideways.

    The \emph{Depth-First} approach traverses downwards through each branch of a (sub)tree
    before backtracking. It is implemented using a \emph{stack}, which will enable 
    the use of recursion -- since each subtree are themselves binary trees. Compare
    this to the \emph{Breadth-Frist} approach which visits each node sideways,
    level-by-level, requiring a \emph{queue} for its implementation.
    
    There are a few different orders within the Depth First approach. 
    
    
    % An Inorder traversal produces the infix representation of the expression.
    % A Postorder traversal produces the postfix representation of the expression. (RPN)
    % A Preorder traversal produces a representation that is the same as the way
    % that the programming language Lisp processes arithmetic expressions. (PREFIX)
    \subsubsection*{Preorder traversal}
    % Traverses the root node first, then the left and right subtrees respectively.
    % % #### Tror det var den jag de facto byggde!!! 
    % NLR (Node Left Right)
    
    Preorder (DLR): Read the data of the node, then visit the left
    subtree/nodes, followed by the right subtree/nodes.

    HÄR KAN DU VISA DIN PRINT TRÄD FUNKTION
    % 1. Traverse the root node
    % 2. Traverse the left subtree
    % 3. Traverse the right subtree

    % It is useful when searching for an element within a BST

    % A preorder traversal is used to create a copy of the tree and is also used To
    % get prefix expression on a expression tree

    % % a pre-order traversal on a topologically ordered tree is useful for listing
    % % the dependency before the dependants. Topological ordering has use cases in
    % % dynamic linking of programs and resolving dependencies as well as processor
    % % scheduling.

    \begin{figure}[H]
        \centering
    \begin{minted}[
        % label=codeexample, 
        linenos, 
        % breaklines, 
        % frame = single, 
        fontsize=\footnotesize]{julia}
# Pretty-Printing
function show(io::IO, node::BTNode{K,V}) where {K,V}
    print(" key: ", node.key, " => value: ", node.value)
end

# Tabbing
print_tabs(numtabs::Int64) = for i = 1:numtabs print("  ") end

# THe wrapped method to the print tree function
function print_tree(tree::BTree{K,V}, 
    node::Union{BTNode{K, V}, Nothing} = tree.root
    ) where {K, V}

    node === nothing && println("---<empty>---") && return
    println("ROOT")
    print_tree(node, 0)
    println("DONE")
end

# recursive method to print tree function
function print_tree(
    node::Union{BTNode{K,V},Nothing}, 
    level::Int64
    ) where {K,V}

    if node === nothing
        print_tabs(level)
        println("---<empty>---")
        return
    end

    print_tabs(level)
    println(node)
    print_tabs(level)
    println("LEFT")
    print_tree(node.left, (level += 1))

    print_tabs(level)
    println("RIGHT")
    print_tree(node.right, (level += 1))

    print_tabs(level)
    println("LEAF")
end
    \end{minted}
    \caption{xxxxx}
    \label{code:print}
    \end{figure}

    \subsection*{The Traversal Algorithms}
    \label{sec:treetraversal}

    \subsubsection*{Inorder traversal}
    % Left Node Right
    Inorder (LDR): Visit the left subtree/nodes, then read the data of the node,
    and finally visit the right subtree/nodes.

    % Traverse the left subtree 
    % Traverse the root node
    % Traverse the right subtree

    % Resultatet blir att nycklarna skrivs ut i ordning!!

    % Inorder traversal gives nodes (get keys) in ascending order. A variation is RNL where an inorder is reversed, 
    % in order to get keys in descending order.


    \subsubsection*{Postorder traversal}
    % LRN (Left, Right, Node)
    Postorder (LRD): Visit the left subtree/nodes, then visit the right
    subtree/nodes, and finally read the data of the node.
    
    % Används vid GC och RPN!!!!!
    % Used to delete the tree and is also useful to get the postfix expression of
    % an expression tree (alltså RPN)
    % % Post-order traversal was also quite popular at some point as part of a
    % % method of expression evaluation known as Reverse Polish Notation (RPN) or
    % % postfix notation. RPN is used to express mathematical formulas in such a way
    % % that there is no need to use brackets to keep the order of precedence. For
    % % example, using RPN we could represent this mathematical expression:
    

    % % +towards the
    % % left-most node. Once it reaches null, it traverses up the tree
    % % back to the root and then to the right-most node. If it hits a node with
    % % children, it will traverse through that node’s children from left to right
    % % and then continue rightwards.

    \section*{Results}
    \label{sec:results}
    \section*{Discussion}
    \label{sec:discussion}
\end{document}